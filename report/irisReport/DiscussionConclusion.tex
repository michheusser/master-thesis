%!TEX root = thesis.tex
\section{Conclusion}

The original aim of this piece of research was to find the theoretical bridge between the two main areas of helical microrobots. The magnetic properties of them (in other words, how they magnetize under an applied field) and the fluid dynamics of the helical shape. Although, as it was shown in the Introduction, there is plenty of research that addresses both areas, there are few that really try to attack both at the same time trying to find an integral solution that allows to optimize the output (the movement of the robot itself) with its input, the rotating magnetic field.\\

Although it might seem that there is enough research on the magnetic properties of the helices, diving more in depth in the topic let us realize that there are some important gaps in the theory that have to be cleared out in order to have a clear view on how to tackle the original problem.\\

First of all, most of the recent research on the area of magnetism used tools and assumptions like the demagnetization matrix without questioning its real definition in the theory of magnetism. Therefore the first approach taken in this work, was to develop a theory that starts with the basic principles of electromagnetism and construct the solutions in order to arrive to the formal definition of the demagnetization matrix. This let us realize the first important part of this work, which is the differentiation of low fields and high fields. Most applications on microrobots involve low applied fields, but the few formal definitions of the demagnetization fields assumed magnetic saturation, which is only achieved with low field. The challenge then, was to prove, that there actually exists a demagnetization matrix for low fields. Although this might seem like an easy assumption, it was only possible to prove until the linearity properties of the derived integral equation were properly analyzed. Having set up a robust and clear theory for the demagnetization matrix for both separate cases, the low field case, and the saturated (high field) case we were able to dig more in depth in the practical calculations of these, checking consistency with existing research (for example with known shapes).\\

We already knew from existing research, that the demagnetization matrix for high fields is a 3D matrix surface integral to solve. The practical implementation of the evaluation of this integral was a numerical challenge that needed various mathematical manipulations (e.g. the conversion to a volume integral) to find the most suitable way. In this area we found that the best way of solving this numerically is performing a surface integral, avoiding this way the complications of the highly singular matrix kernel in the volume integral. The problem with this method is that, for each point in the helix, one has evaluate the surface nine 3D surface integrals, which makes the averaged demagnetization field a task which needs highly performing computers to perform. Optimizing the calculation of the averaged demagnetization matrix is a task to look more in depth in further research. Besides the challenges and the calculation intensity of this integrals, we were able to develop an integral method that showed close to equal results than the ones obtained through simulations\\

In the case of low fields, there is no research found on the calculation of demagnetization matrices through integral methods. The theory let us know that the demagnetization matrix (similar to the high fields case) is a 3D integral surface integral problem, this time one has to find the solution to an integral equation, for each point in the material. This poses a major calculation problem. We tried to dive deeper in the solving of this problem, specifically the solving of the integral equation. Unfortunately, the integrals used in the equation have a very complex and highly singular structure, that made the problem specially challenging. Through the experimentation with multiple methods, including the one we developed ourselves (shown previously) and unsatisfactory results, we decided to leave this problem for further research, since it needs sophisticated mathematical tools and methods specifically tailored for magnetic problems. For the theoretical results we had to rely on the simulations performed with FEM.\\

One of the most important parts in this work was doing FEM simulations in COMSOL. We validated some shapes were the demagnetization matrices were known algebraic expressions and we were able to confirm the correctness of the simulation environment. This allowed us to assume always the simulation environment as correct and as a reference to the closest match of the theoretical solution. Since the software was very intuitive and mature, it wasn't hard to deal with the different tools put to disposition and exploit the functionalities, also with the help of Matlab. The main problems encountered in the simulations were at the simulation of non-linear magnetic behavior, which led to convergence errors in the solvers at first. The other problem encountered was the meshing of the half coated since it is a very thin shape. Both problems were solved with the help of the support team of COMSOL. Besides that, finding a way of modeling in 3D the half coated shape for coiled indexes (H1.5 and below) was a major challenge that, even after several attempts, wasn't successful due to meshing problems. This will be left for further research.\\

Out of the demagnetization matrices we calculated for the helices H1 to H10 (low fields and high fields), we could calculate the misalignment angles for the helices and compare them to real measured data. We saw that we could clearly capture the trend of the curve. Taking in consideration practical factors like the influence of the wall (where the measurements were done) could help give a more realistic comparison.\\

After having settled clear the theoretical differences of the both cases, the saturated case and the low field case and having had the simulations as a reference point, a connection to the real case was needed. The need of real VSM measurements and the comparison to its simulated counterpart was to be analyzed to give the work done a deeper meaning. We manually coiled up 4 different macro helices and measured its (averaged) magnetization with the help of the VSM. We first used a calibration nickel sphere to have the real material properties of nickel. Our first realization was that, unlike most of the literature pieces that have dealt with this topic, the intrinsic susceptibility wasn't as high as we had assumed from the beginning on (the experiments showed that $\chi_m = 24$). We then created a simulation environment with this new intrinsic susceptibility.\\

The simulation part was not a big challenge since it was basically the same as it was used for the microhelices. Since the material we're using is polycrystalline and we're dealing with sizes much bigger than the sizes of the magnetic domains, therefore the material properties are the same in the macro case as in the micro case. The real challenge here was the evaluation of the measurements. It seems that we arrived at key questions that have not been asked often in the area of microhelices. We are dealing with a ferromagnetic helices with very complex properties that have to be simplified in meaningful ways. The magnetization of the helices happens in a hysteretic loop, that, although being thin, magnetizes in a different way the first time it magnetizes as it is after it is inside the loop. This is the reason why the question of how a helix magnetizes in low fields has no clear answer. Further research should put in context the environment in which the helices are going to be magnetized.In our case, we assumed that the helices are already in the magnetization loop (it is not the first time they magnetize) and therefore the first magnetization part was left unexamined and the loop was averaged such that only one clear function remains.\\

Another big challenge, encountered when analyzing the measurements, was to define the saturation point of the helices. Unlike the sphere, where this is a perfectly linear behavior that reaches saturation at a specific point and then stays that way (in theory, of course), the helices, in practice don't have a clear saturation part and therefore an assumption has to be made in terms of percentage of saturation achieved, which was taken from the measurements of the calibration sphere.\\

The results obtained from the simulation and the VSM measurements were satisfactory and gave a clear insight of correctness in terms of the easy axes and the qualitative diagonal values of the demagnetization matrices between each other. Of course a bold assumption was to take the same intrinsic susceptibility for the sphere as for the helices. In the future, a sphere with exactly the same material properties (composition and material) should be used for the intrinsic susceptibility. Further research should analyze more in depth a meaningful way of simplifying the measurements and other possible ways of defining the saturation point, which is crucial to get the correct values for the demagnetization matrices in high fields.\\

This thesis should serve as an important part for further research to complement the magnetic behavior of the helices and complement the theories that include both the magnetism theory and the fluid dynamical theory to optimize the response of applied fields towards the dynamic and kinetic properties of the helices. In the search of the optimal helix shape for our purposes it is still an early stage to chose an optimal shape before combining it with the fluid dynamical relations. We still saw an optimizable behavior in the demagnetization fields since the curves had global maxima and minima. Further research will have to analyze the behavior of the helices in a free unconstrained environment in presence of applied fields since in this thesis only constrained environments were considered. This should complement the search for an optimizable theory to find the optimal alignment and its comparison to the experiments.
