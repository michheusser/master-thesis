\section*{Abstract}

This thesis starts by doing an extensive summary of existing and trending technologies in the field of microrobots for life sciences and puts emphasis on the benefits and potential of magnetic helical microrobots. It then proceeds to do an extensive mathematical analysis of the physics of magnetism and derives the formal definition of the very often used demagnetization matrices all the way from the Maxwell equations. From this formulation it is proven that, for soft-magnetic materials, one has to do the distinction between low and high fields, since the mathematics behind it are vastly different. The general high field case, where the magnetic body is saturated deals with surface integrals in three dimensions whereas the low field case deals with surface integral equations in three dimensions. Since the numerical solution to these problems are highly complex, magnetic simulations using FEM were used to assess the correctness of the numerical implementations of the solutions, as well as giving solutions where the numerical implementation was not possible. To be able to compare the real magnetization case of helices, nickel macrohelices were manually coiled and their magnetic properties were analyzed using a VSM setup which was then compared to the FEM simulated model.
